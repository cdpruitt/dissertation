\epigraph{``The grandest discoveries of science have been but the rewards of
    accurate measurement and patient long-continued labour in the minute
sifting of numerical results.''}{William Thompson, \nth{1} Baron Kelvin}

\section{Relevant Models of the Atomic Nucleus: Overview}
What do the nucleons do in the nucleus? Where are the neutrons and the protons
with respect to each other?
- What is needed to expand beyond current workhorse models (Droplet Model,
Shell Model, Optical Models) and connect nuclear reactions and structure

- What are the neutron scattering data for cornerstone nuclides (magic Z,N) and
what do they tell us about the isoscalar/isovector terms of the optical
potential and where the nucleons go/what they do in the nucleus?

- What types of experimental data are most essential for an improved understanding of nuclear
reactions and structure?

To unravel the nuclear many-body problem, we can
employ models that incorporate essential details while ignoring specifics with
little relevance to the observables we care about.

Successful models possess both accuracy at reproducing extant experimental data
and predictive power for as-yet unmeasured experimental data (far harder). For parametric models
with many tunable parameters, these criteria pull in oppositive directions: increasing the number
and acceptable range of model parameters often helps to reproduce experimental data but may
jeopardize predictive power if new parameters are not connected to the underlying physics.

Several workhorse models relevant to this dissertation are presented below. Their successes,
failures, and regimes of validity are briefly discussed, with extra attention paid
to each model's confrontation with challenging experimental data.

\subsection{Liquid Drop Model}

The Liquid Drop Model (LDM) describes nuclei as drops of ideal nuclear fluid and
has been successfully employed since the earliest days of nuclear science to
describe nuclear masses and other ground-state properties. The binding of each
nucleus is approximated by five physically-intuitive terms appropriate for
a droplet of nuclear matter:

\begin{equation} \label{LDM}
    BE(Z, N) = Vol + Surface + Coulomb + Asymmetry + Spin-Orbit
\end{equation}

a volume term that describes "bulk" binding that would be experienced in an
infinite sea of nuclear matter, or by completely surrounded nucleons in the core of the nucleus,

a surface term that incorporates the finite size of a nucleus (i.e., it is a
drop, not an ocean), equivalent to surface tension,

a coulomb term that incorporates the electric repulsion experienced by protons
constrained in close proximity to each other inside the drop,

an asymmetry term representing the relative chemical potential of neutrons and
protons as a function of their relative population (which can be re-balanced by
beta-decay),

and a spin-orbit term responsible for modeling the interaction of the intrinsic spin
of a nucleus and its constituent nucleons' angular orbital momentum, which is
a much stronger effect in nuclear binding than in the atomic case

The free parameters in each term can be fitted to the hundreds of well-measured nuclear masses
across the chart of nuclides. These five simple terms are quite successful
in describing masses and radii of spherical nuclei, leading early nuclear
scientists to expect that shell structure was less important in the nuclear
many-body problem than in the atomic one. In this ansatz, the quantum
nature of constituent nucleons is completely ignored, so the LDM is  
unsuitable for extracting wavefunction information or predicting scattering
cross sections.

Extensions to the LDM introduce a series of higher-order terms allowing for coupling
between the five components of \ref{LDM}, each of which that have been expanded about two
fundamental independent quantities, the nucleon density excess $\epsilon$
and neutron density excess $\delta$:

\begin{equation}
    \epsilon = -\frac{1}{3}\frac{(\rho - \rho_{0})}{\rho_{0}}
\end{equation}

\begin{equation}
    \delta = \frac{\rho_{n}-\rho{z}}{\rho}
\end{equation}

where $\rho_{0}$ is the saturation density, 0.16 nucleons fm$^{-3}$. Relevant quantum
effects, such as changes in level densities near shell closures
that affect the observed masses, can be incorporated through a series of corrections
that approximate shell effects, nucleon-nucleon pairing, and geometric deformation.
One such model, presented in The Droplet Model of Atomic Nuclei
\cite{MyersAndSwiatecki}, deploys 9 
independent coefficients to describe spherical droplets and 6 additional
coefficients to accommodate non-spherical effects. This expanded scope
can successfully recover the degree of ground state deformation in non-spherical
nuclei and fission barriers.

By the 1980s, it was clear that the isotope shift in the nuclear RMS
charge radius, shown for the even-A Sn isotopes in Fig.
\ref{SnIsotopeShift} \ref{Anselment1986, Otten1989},
is critically connected to difference in RMS radius 
between the neutron and proton distributions, commonly referred to as the "neutron skin".
In a Droplet Model picture, the slope of the isotope shift is linked to the symmetry energy $J$, 
density dependence of the symmetry energy
$L$, and surface stiffness coefficient $Q$, the same variables that determine
neutron skin thickness \cite{Myers1969, MyersAndSwiatecki, Berdichevsky1988}.
Without addition experimental constraints, the $J$, $L$, and $Q$ parameters form an
underdetermined system from which unique values cannot be recovered, even if the
consensus value of $J \approx$ 30 MeV is used to reduce the parameter space.
As was already pointed out 50 years ago by Myers \cite{Myers1969},
the single-particle configuration of a given nucleus may have a significant
effect on the formation of the 

\begin{figure}
    \includegraphics[width=0.8\textwidth]{figures/SnIsotopeRMSRadii.png}
    \caption{RMS charge radii in Sn nuclei. Data from Anselment et al. (1986)}
    \label{SnIsotopeShift}
\end{figure}

\subsection{Mean-field and beyond-mean-field models}
Mean field models begin with a motivating assumption: that nucleons
experience the nuclear environment independently, as an average
potential generated by all other nucleons, smeared out over nuclear volume.
The assumption of independent nucleon motion may seem dubious, given the
crowded environment of the nucleus and immense strength of the nucleon-nucleon forces,
but the Pauli exclusion principle provides some justification.
Confined by this mean field, nucleons obey a nuclear aufbau, filling orthogonal states with
quantum numbers N, L, and J, as electrons do in the atomic case, resulting in
shell structure. From a mean field consisting of a central
potential and a spin-orbit potential, as in the seminal work of Goeppert-Mayer
and Jensen \cite{GoeppertMayer1955}, the basic ground-state quantum properties of most nuclei
(spins, parities, magnetic moments) are recovered. In addition, nucleons can be
excited into higher (but still bound) states and the low-lying excitation
spectra predicted. The consequences of
shell structure are clearly present in experimental data, including increased particle separation 
energies and decreased nuclear radii at shell closures, directly analagous to
the atomic ionization energies and radii in the noble gases. The independent
treatment of nucleons is most valid near shell closures,
where the level density is reduced, and near beta-stability, where coupling to
the asymptotically-free states of the continuum is least important.

Typically, the central average potential is considered as only a starting point for a perturbative
expansion that collects the residual nucleon-nucleon interactions associated with
correlated behavior, such as clustering and oscillatory modes. Coupled excitations of
two or more nucleons to higher orbitals and relativistic effects may be included to 
accommodate the experimental phenomena under investigation.
In light systems, where the number of nucleons is not too large, each nucleon is
allowed to participate in excitations into the valence space of the model
(``no-core shell model"). As the system size increases, the configuration space grows
combinatorially until calculations become prohibitively expensive for a no-core
shell model approach. To make these nuclei tractable, a variety of approaches
are employed, including simply restricting the valence space and prohibiting deeply-bound nucleons 
from participating in excitations, ideally with relevant physics still intact.

\begin{figure}
    \includegraphics[width=0.8\textwidth]{figures/ShellModel.png}
    \caption{Nuclear shell model }
    \label{ShellModel}
\end{figure}

Mean-field challenges: Adding coupling to the continuum to describe systems far from stability,
astrophysical reactions.
- failure to recover depletion/spectral peak broadening. In what regimes are nucleons the best
choice of degree-of-freedom?

Discuss beyond-the-mean field:
- mention chiral-EFT
- Av18 and microscopic potentials
- Connect to chiral effective field theory work for calculating optical potentials
and examining isovector component of potential, especially w/r/t to exotic
nuclei near driplines. Nucleons are still degree of freedom, but allowed to exchange effective pions
that include the quark-quark interactions that comprise the strong nuclear force.
- Lattice QCD: nucleons are no longer the degree of freedom, but nuclear physics
is almost completely out of reach (going beyond He-4 currently impossible).

\subsection{Density Functional Theory}
- Fission
- Heavy systems (no lighter than Ca40)
- Omit this section entirely? Not relevant to DOM.

\subsection{Optical Models}
The initial motivation for all the models presented thus far has been recovery
of structural observables like nuclear masses, radii, low-lying excitations, and magnetic moments. 
A comprehensive understanding of the nucleus must also say something about what
nuclei \textit{do}, as in nucleus-nucleon scattering experiments and astrophysical reactions.
The nuclear optical model (OM) was developed to this end and continues to be an
widely-used tool for generating nucleon, $\alpha$, and heavy ion scattering
cross sections, though it has less to say about nuclear structure. Basic
motivations for OMs, plus references to successful OM potential parameterizations,
are summarized here.

Due to the magnitude of the strong nuclear force, it might be expected that 
the interaction of an incident neutron on a nucleus should be strongly
absorptive, with only a small contribution from elastic scattering. Thus, the
earliest model for neutron scattering describes the nucleus as a constant-density
sphere that interacts strongly with incident neutrons approaching within a nuclear radius
\cite{Feshbach1949}. In this ``strongly-absorbing sphere" (SAS) picture devoid of nuclear structure
or shell effects, \tot\ depends only on size scaling of the interacting bodies:
\begin{equation} \label{SASAbsolute}
    \sigma_{tot}(E) = 2\pi(R + \lambdabar)^{2}
\end{equation}
where $R=r_{0}A^{\frac{1}{3}}$ and $\lambdabar$ is the reduced wavelength
of the incident neutron in the center of mass \cite{Fernbach1949, Satchler1980}. 

As neutron scattering experiments expanded to higher energies in the 1950s, surprising 
neutron \tot\ data emerged that challenged this picture. In Fig.
\ref{SASphereVsExperiment}, experimental neutron \tot\ data are shown from 2-500
MeV for nuclides from A=12 to A=208 \cite{Finlay1993, Schwartz1974, Poenitz1983, Abfalterer2000, 
Abfalterer2001}. Predictions for \tot\ given by Eq. \ref{SASAbsolute} are shown as thin dashed 
lines for each nucleus. Regular oscillations about the SAS model are clearly
visible, as is the trend for the oscillation maxima/minima to shift to \textit{higher} energies as 
A is increased. At low energies, resonance structures are visible especially for light nuclides 
where the density of states is smallest. Note that at higher neutron energies, the experimental
cross sections drop below those predicted by the SAS model, illustrating
a relative increase in nuclear transparency.

 \ref{SASphereVsExperiment}.

\begin{figure}
    \includegraphics[scale=0.4]{figures/SASphereVsExperiment.png}
    \caption{Experimental neutron \tot\ data and strongly-absorbing-sphere predictions}
    \label{SASphereVsExperiment}
\end{figure}

These hallmark oscillations in the neutron \tot\ can be explained as the result
of a phase shift between 
neutron waves passing around the nucleus (unshifted) and waves passing
through the the nucleus, where they experience refraction
(illustrated in Fig. \ref{RamsauerPhaseShiftFigure}). This explanation was termed the ``nuclear 
Ramsauer effect" by Peterson \cite{Peterson1962}, based on the analagous effect seen in 
electron scattering on noble gases.

Following Angeli \cite{Angeli1970}, these considerations can be incorporated by
imbuing the strongly-absorbing sphere relations (equation \ref{SASAbsolute}) with an additional sinusoidal term:
\begin{equation} \label{OscillatoryModel}
    \tot = 2\pi (R+\lambdabar)[1 - \rho \cos(\delta)]
\end{equation}
where $\rho = e^{-\operatorname{Im}(\Delta)}$, and $\delta =
\operatorname{Re}(\Delta)$, with $\Delta$ the phase difference between the wave traveling
around and traveling through the nucleus. Thus, the amplitude of the oscillations provides the 
inelastic phase shift and the period of oscillation provides the elastic phase shift.
As can be seen from Eq. \ref{OscillatoryModel}, the large magnitude of the oscillations means that 
inelastic scattering (from $\operatorname{Im}(\Delta)$) accounts for only a small fraction of the 
total cross section, in turn implying a much larger mean free path for neutrons through the nucleus 
than might otherwise be expected in the absence of Pauli blocking \cite{Mohr1955}.

If we approximate the nucleus with a
spherical potential of radius $R$ and depth $U$, the total phase shift $\delta$ is:
\begin{equation} \label{phaseShift}
    \delta =
    \frac{\overline{C}\left(\left[{\frac{E+U}{E}}\right]^{\frac{1}{2}}-1\right)}{\lambdabar}
\end{equation}
where $\overline{C} = \frac{4}{3}R$ is the average chord length through the
sphere \cite{Angeli1970}. Rearranging Eq. \ref{phaseShift} in terms of A and E and
discarding leading constants yields:
\begin{equation}
    \delta \propto A^{\frac{1}{3}}\times\left(\sqrt{E+U}-\sqrt{E}\right)
\end{equation}
This form reveals an important relation: as A is increased, to maintain constant 
phase $\delta$, E must also increase \cite{Satchler1980, Peterson1962}. 
This is contrary to a typical resonance condition where an integer number of wavelengths
are fit inside a potential; in that case, to maintain constant phase as A is increased,
E must be decreased. Thus these \tot\ oscillations have been referred to as
``anti-resonances" or ``echoes" \cite{Satchler1980, McVoy1967}.

Thus a new type of nuclear model is at hand: by replacing the intractable many-body problem
of the target nucleus by a complex, refractive potential, both elastic scattering (from the real 
part of the potential) and inelastic scattering (from the imaginary part) could be neatly 
explained. The existing mathematical machinery for calculating scattering of light from
refractive materials could be repurposed for nuclear scattering, giving birth to
the ``optical model" of the nucleus, \cite{Feshbach1958, McVoy1967}. In
practice, the potential is typically parameterized with a series of
terms centered on the nuclear surface and nuclear volume,
corresponding to differing physics thought to be important for these areas (much
as in the Droplet Model).

\begin{figure}
    \hfill\begin{minipage}{0.5\textwidth}\centering
        \includegraphics[scale=0.2]{figures/phaseShiftStillsFigure.png}
        \caption[A illustration of the nuclear Ramsauer effect]{
            A neutron wave train (series of
            blue lines) impinges from the left on a real Woods-Saxon
            potential centered at the origin (diffuse red circle). The potential
            refracts the neutron wave,
            retarding the phase of the wavefront as it passes through the
            potential. After escaping the potential, a phase difference $\Delta$ between
            the wave component passing \textit{around} and \textit{through the center}
            of the potential persists, resulting in scattering.
            For the leading wavefront in the wave train, $\Delta$ is indicated in
            the top right-hand corner of each panel. A differential version of
            Eq. \ref{phaseShift} is used to
            calculate the phase shift for each step. In this figure, the neutron
            energy $E_{n}$ = 14 MeV and nuclear mass $A$ = 25. For the Woods-Saxon potential,
            we used a potential depth $U$ = 42.8 MeV (following Angeli's analysis
            of \tot\ data at 14 MeV \cite{Angeli1970}), with nuclear radius $R = 
            r_{0}A^{\frac{1}{3}}$, $r_{0}$ = 1.4 fm, and a diffuseness parameter
            $a$ of 0.5 fm.
        }
        \label{RamsauerPhaseShiftFigure}
    \end{minipage}
\end{figure}

Global OMs have been developed to simultaneously reproduce single nucleon, heavy ion,
and other hadron scattering data on targets across the chart of nuclides up to several
hundred MeV \cite{CH89, KoningDelaroche}. In addition to isoscalar-dependent terms
in OM potentials that treat protons
and neutrons identically, isovector and isotensor terms that break isospin
symmetry have been included to describe proton and neutron scattering with the
same potential. However, despite the excellent agreement with experiment, OMs 
involve the interaction of many sometimes-opaque parameters with many incident
partial waves, complicating intuitive understanding of the underlying physics at work.
OMs are unabashedly phenomenogical so a wide variety of
experimental data types and energies are required to constrain the potential.
Where data are absent, model predictions are poor. A long-standing issue has
been a poor understanding of the isovector dependence of the real and
imaginary parts of optical potentials \cite{Holt16}, a consequence of the
difficulty of neutron scattering experiments. As experimental facilities like
the Facility for Radioactive Isotope Beams (FRIB) come online and produce
extremely asymmetric nuclei, knowing the asymmetry-dependence of optical
potentials becomes paramount. Providing such experimental constraints for optical
potentials, in particular for the Dispersive Optical Model (DOM), is a primary 
motivation for the isotopically-resolved neutron \tot\ and \el\ work presented in this dissertation.
In chapter \ref{DOMFormalism}, I present an overview of the DOM, a descendent of traditional OMs 
that extends the potential to negative energies to generate nuclear structure.

%"The magnitude and energy dependence of the real isovector part of the
%optical potential are poorly constrained by experiment." \cite{Holt16}

Other topics that could be included:
- Hodgson's optical model review in 1971
- Microscopic (folds interaction over nucleon density)
- Phenomenological (no folding - use bulk)
- can largely follow Bob and Wim's review paper
- first formulated in 1949 to describe neutron cross section data
- separate proton and neutron global optical potentials in the 60s (Becchetti
and Greenlees)
- CH89 includes additional neutron scattering data on isotopic targets
- Lane potential 
- Isovector versus isoscalar. Charged pion exchange mediates the isovector forces;
other nucleon-nucleon interactions mediate the isoscalar forces.

\section{The Landscape of Experimental Nuclear Data}

\subsection{Elastic nucleon scattering}
Elastic nucleon scattering cross sections comprise the most extensively-measured
sector of experimental scattering data.

\subsection{Inelastic nucleon scattering}

\begin{figure}
    \includegraphics[width=0.8\textwidth]{figures/TCSChart.png}
    \caption{Landscape of existing neutron \tot\ data (per the EXFOR database)}
    \label{TCSChart}
\end{figure}

\begin{figure}
    \includegraphics[width=0.8\textwidth]{figures/RCSChart.png}
    \caption{Landscape of existing proton \rxn\ data (per the EXFOR database)}
    \label{RCSChart}
\end{figure}

Neutron scattering is a direct, Coulomb-insensitive tool for probing the nuclear
environment. The simplest measurement of neutron interaction with a nucleus,
the neutron total cross section \tot, provides fundamental information about
nuclear size and the ratio of elastic-to-inelastic components of nucleon 
scattering. Additionally, \tot\ data are sensitive to a variey of nuclear
properties of great interest including the neutron skin of neutron-rich nuclei
\cite{Mahzoon2017} and thus the density dependence of the symmetry energy $L$,
essential for an accurate neutron star equation-of-state (EOS)
\cite{Fattoyev2012, Vinas2014, Brown2000}.

To compare \tot\ for two different targets of masses A and A', the relative
difference \totRD\ is useful:
\begin{equation} \label{SASRelDiff}
    \begin{split}
        \totRD & \equiv
    \frac{\sigma_{A}-\sigma_{A'}}{\sigma_{A}+\sigma_{A'}} \\
    & =
    \frac{r_{0}^{2}(A^{\frac{2}{3}}-A'^{\frac{2}{3}}) +
    2\lambdabar r_{0}(A^{\frac{1}{3}}-A'^{\frac{1}{3}})}
    {r_{0}^{2}(A^{\frac{2}{3}}+A'^{\frac{2}{3}}) +
    2\lambdabar r_{0}(A^{\frac{1}{3}}+A'^{\frac{1}{3}}) + 2\lambdabar^{2}}
    \end{split}
\end{equation}

ref{SASphereVsExperiment}).

By scattering secondary radioactive beams off of hydrogen targets in inverse
kinematics, proton-scattering experiments are possible even on highly unstable
nuclides. In contrast, because neutrons themselves must be generated as a
secondary radioactive beam, neutron-scattering experiments are restricted to
normal kinematics and \tot\ measurements are possible only for relatively stable
nuclides that can be formed into a target. At present, \tot\ measurements above
the resonance region on nuclides with short half-lives (shorter than the timescale of
days) are technically infeasible for this reason, though a handful have been carried out on
samples with half-lives in the tens to thousands of years \cite{Poenitz1983,
Phillips1980, Foster1971}.

Traditionally, \tot\ measurements have relied on analog techniques for recording
events, techniques that suffer from a large per-event deadtime of
up to several $\upmu$s. Thus for a typical intermediate-energy \tot\ measurement
with dozens or hundreds of energy bins, achieving statistical uncertainty at the
level of 1\% requires a thick sample to attenuate a sizable fraction of the
incident neutron flux. For cross sections in the 1-10 barn range, this means
sample masses of tens of grams \cite{Finlay1993, Abfalterer2001}.
Producing an isotopically-enriched sample of this size is often
prohibitively expensive; indeed, a literature search for isotopically-resolved
\tot\ measurements reveals a paucity of data from 1-300 MeV, even for
closed-shell isotopes of special importance like $^{3,4}$He, $^{64}$Ni, and
$^{204}$Pb (see Table \ref{IsotopicCrossSectionTable}).

\begin{table}[ht]
    \caption[Selected results from a literature study of isotopically-
        resolved \tot\ data using the EXFOR database \cite{EXFORDatabase}]
        {Selected results from a literature study of isotopically-
        resolved \tot\ data using the EXFOR database \cite{EXFORDatabase}
        For the heaviest and lightest stable nuclides in each closed shell in Z, all
        datasets falling at least partially within 1-500 MeV are shown. For elements
        whose natural abundance is $>$90\% of a single isotope (e.g.,
        96.9\% of $^{\text{nat}}$Ca is $^{40}$Ca), \tot\ data on the natural
        target was included as ``isotopic".}
    \label{IsotopicCrossSectionTable}
    \begin{center}
        \begin{tabular}{ c c c c }
            \hline
            Isotope & Nat. Abund. & Energy Range & Reference\\
                    & [\%] & [MeV] & \\

            \hline

            $^{3}$He & $2\times 10^{-4}\%$ & $1.5 - 40$ & \cite{Haesner1983}\\
            $^{4}$He & $>99.9\%$ & $0.7-30$ & \cite{Goulding1973}\\
            & & $2-40$ & \cite{Haesner1983}\\
            & & $77-151$ & \cite{Measday1966}\\

            $^{16}$O & $99.8\%$ & $0.2-49$ & \cite{Perey1972}\\
            & & $5-600$ & \cite{Finlay1993}\\

            $^{18}$O & $0.20\%$ & $0.1-2.5$ & \cite{Vaughn1965}\\
            & & $2.5-19$ & \cite{Salisbury1965}\\

            $^{40}$Ca & $96.9\%$ & $<0.1-6.4$ & \cite{Johnson1973}\\
            & & $5.3-560$ & \cite{Abfalterer2001}\\

            $^{48}$Ca & $0.187\%$ & $0.6-5.2$ & \cite{Harvey1985}\\
            & & $12-276$ & \cite{Shane2010}\\

            $^{58}$Ni & $68.1\%$ & $<0.1-68$ & \cite{Perey1993}\\

            $^{64}$Ni & $0.926\%$ & $14.1$ & \cite{Dukarevich1967}\\

            $^{112}$Sn & $0.97\%$ & $<0.1-1.4$ & \cite{Timokhov1989}\\
            & & $14.1$ & \cite{Dukarevich1967}\\

            $^{124}$Sn & $5.79\%$ & $0.3-5.0$ & \cite{Harper1982}\\
            & & $5.1-26$ & \cite{Rapaport1980}\\

            $^{204}$Pb & $1.4\%$ & $<0.1-27$ & \cite{Carlton2003}\\

            $^{208}$Pb & $52.4\%$ & $<0.1 - 695$ & \cite{Harvey1999}\\
            & & $5-600$ & \cite{Finlay1993}\\

            \hline
        \end{tabular}
    \end{center}
\end{table}

New neutron scattering data (both \tot and \el) on isotopic targets, grist for the DOM mill and the 
central result of this dissertation, are presented in Chapters \ref{TCSAnalysis, ECSAnalysis}.

\subsection{Electron scattering and laser spectroscopy}
K. Minamisono

\subsection{Quasi-free scattering}
Louk Lapikas

\subsection{Nuclear Masses, Matter Radii, and Charge Radii}
Connect to Vinas, Chuck Horowitz, and J. Piekarewicz's neutron star EOS work.

\section{Motivation, Scope, and Dissertation Outline}
Improved isotopically-resolved neutron scattering data are an essential ingredient
for better nuclear reaction models and to test the asymmetry-dependence of the
nuclear potential. The results of our campaign to produce these valuable
\tot\ and \el\ data sets on cornerstone nuclei form the backbone of this dissertation.
In addition to these experimental results, a suite of Dispersive Optical Model (DOM) analyses that 
incorporate these new data is presented. From the DOM potentials, a variety of asymmetry-dependent 
nuclear-structure quantities, including neutron skins and relative momentum
content, are extracted.

An overview of neutron \tot\ experimental considerations, the details of our 
\tot\ experiment, and analysis for our isotopically-resolved \tot\ measurements
on \oSixEight, \niEightFour, \rhThree, and \snTwelveFour are detailed in 
Chapters \ref{TCSExperiment} and \ref{TCSAnalysis}. Similarly, our elastic scattering measurements 
on \snTwelveFour are presented in Chapters \ref{ECSExperiment} and \ref{ECSAnalysis}. A brief 
summary of the Dispersive Optical Model formalism is
given in Chapter \ref{DOMFormalism} to contextualize the results from our DOM fits of \oSixEight, 
\caAughtEight, \niEightFour, \snTwelveFour, and \pbEight presented in Chapter \ref{DOMResults}. 
A complete listing of the experimental data used in the DOM analyses, the
parameter values of the DOM potentials, and figures showing the 
results of the DOM fits are provided in Appendices \ref{DOMDataSets,
DOMParameters, DOMFits}. 
