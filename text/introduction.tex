%\begin{figure}
%  \begin{center}
%  \includegraphics[width=0.8\textwidth]{Chart_of_Nuclides.pdf}
%  \caption{Figure caption}
%  \label{chart}
%  \end{center}
%\end{figure}

\section{Models of the atomic nucleus}

To unravel the nuclear many-body problem, we can
employ models that incorporate essential details while ignoring specifics with
little relevance to the observables we care about.

Liquid Drop Model

An especially simple and
effective model, the Liquid Drop Model (LDM), describes nuclei as drops of
an ideal nuclear fluid and has been successfully employed for many decades to
describe nuclear masses. The binding forces of each nucleus are modeled by five
physically-intuitive terms appropriate for such a fluid:

a volume term that describes "bulk" binding that would be experienced in an
infinite sea of nuclear matter, or  by completely surrounded nucleons in the core of the nucleus,

a surface term that incorporates the finite size of a nucleus (i.e., it is a
drop, not an ocean), equivalent to surface tension,

a coulomb term that incorporates the electric repulsion experienced by protons
constrained in close proximity to each other inside the drop,

an asymmetry term representing the relative chemical potential of neutrons and
protons as a function of their relative population (which can be re-balanced by
beta-decay),

and a spin-orbit term responsible for modeling the interaction of the intrinsic spin
of a nucleus and its constituent nucleons' angular orbital momentum, which is
a much stronger effect in nuclear binding than in the atomic case

In this model, each term is parameterized and the ensemble can be fitted to
well-measured nuclear masses across the chart of nuclides. Happily, though the detailed
quantum structure is completely ignored, these five terms are quite successful
in describing nuclear masses.

The Independent Particle Model (IPM) provides a complementary approach expressly
focused on the quantum details of the problem. The many-body problem is
approximated by considering each nucleon as moving independently of all other
nucleons in an potential generated by those nucleons. As in the atomic case,
where electrons obey an aufbau principle and populate orbitals defined by the
atomic potential, both protons and neutrons obey a nuclear aufbau, filling
orthogonal states in the nuclear potential. In this model, many fundamental
quantum properties of nuclei are easily understood, (for example, ground state spins and
the appearance of shell closures in direct analogy to the atomic case).
Unfortunately, the assumption of independent nucleon motion that defines the model
also kneecaps its usefulness: in reality, nuclear systems are strongly
correlated and exhibit clustering and collective motion, behaviors at odds with the
the IPM picture.

Additional models (RPA, Density Functional Theories, Optical Models

% \Gls{DOM} for glossary term
% \noindent for no indent

%\begin{comment}
% Woods-Saxon potential 
%\end{comment}
%
%\begin{equation}
%V(r) = \dfrac{-V_0}{1+e^{(r-R)/a}},
%\end{equation}

%\mathbf{J} = math bold-font symbol 'J'
