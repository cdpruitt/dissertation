%\makenoidxglossaries

\makeglossaries

%\GlsXtrEnablePreLocationTag{\textit{~}}{\textit{~}}
%\renewcommand{\GlsXtrFormatLocationList}{\textit}

\newglossaryentry{DOM}
{
    name={DOM},
    description={The Dispersive Optical Model, a phenomenological framework for
    extracting nuclear structure and reaction information from experimental data},
}

\newglossaryentry{optical potential}
{
    name={optical potential},
    description={A complex potential used to approximate the microscopic nuclear many-body problem. Incident nucleons scatter off the potential in analogy to the refraction and absorption of light in optical media.  In both cases, the real component of the potential determines elastic scattering, and the imaginary component determines inelastic scattering},
}

\newglossaryentry{self-energy}
{
    name={nucleon self-energy},
    description={
        A complex, non-local mathematical object that describes the interaction of a
        nucleon with another body (typically a nucleus) via an infinite sum of relevant Feynman 
        diagrams. If the nucleon self-energy is known in a given system, a multitude of other 
        important physics quantities (scattering amplitudes, the mean free path, the level density) 
        can be extracted. The DOM links the optical potential and the nucleon
        self-energy, enabling a phenomenological approach for extracting
        information about the nuclear many-body problem
    },
}

\newglossaryentry{Dyson equation}
{
    name={Dyson equation},
    description={A self-consistent relationship between the dressed propagator $G$, the free
        propagator $G_{0}$, and the irreducible nucleon self-energy, $\Sigma^{*}$ (shown in
        Eq. \ref{DysonEquation}). The equation can be expressed pictorally
        via Feynman diagrams (shown in Fig. \ref{DysonEquationDiagram})},
}

\newglossaryentry{inverse kinematics}
{
    name={inverse kinematics},
    description={An experimental approach where the nucleus under study
        (e.g., $^{14}$O). is bombarded onto a sample containing a typical
        scattering particle (e.g., protons or $\alpha$-particles).
        By reversing the usual kinematics of scattering,
        reactions can be studied on unstable nuclei that
    cannot be made into a fixed target},
}

\newglossaryentry{TUNL}
{
    name={TUNL},
    description={The Triangle Universities Nuclear Laboratory, the site of our neutron \el\ 
        measurements on \snTwelve\ and \snFour},
}

\newglossaryentry{LANSCE}
{
    name={LANSCE},
    description={The Los Alamos Neutron Science Center, the site of our neutron \tot\ measurements on
    \oSixEight, \niEightFour, \rhThree, and \snTwelveFour},
}

\newglossaryentry{WNR}
{
    name={WNR},
    description={Weapons Neutron Research facility at LANSCE, site of a spallation neutron source
    useful for \tot\ measurements},
}

\newglossaryentry{PSD}
{
    name={PSD},
    description={Pulse-shape discrimination, a technique for differentiating between detector events
    caused by neutrons, heavy ions, and $\gamma$ rays},
}

\newglossaryentry{TOF}
{
    name={TOF},
    description={Time-of-flight. Measuring neutron TOF from a pulsed source is a common
    neutron energy determination technique},
}

\newglossaryentry{CFD}
{
    name={CFD},
    description={Constant-fraction discrimination, a technique for determining event timestamps
    independent of the pulse amplitude of the signal},
}

\newglossaryentry{LED}
{
    name={LED},
    description={Leading-edge discrimination, where event timestamps are assigned according to the
    time the leading edge of the signal crosses a fixed threshold},
}

\newglossaryentry{ADC}
{
    name={ADC},
    description={Analog-to-digital converter, a device that digitally records the integral of an 
    incident electrical signal over a window specified by the user},
}

\newglossaryentry{TDC}
{
    name={TDC},
    description={Time-to-digital converter, a device that digitally records the timestamp of an
    incident electrical signal according to a threshold specified by the user},
}

\newglossaryentry{GDG}
{
    name={GDG},
    description={Gate-and-delay generator, a device used to generate
    variable-width logic gates for analog signal processing}
}

\newglossaryentry{propagator}
{
    name={single-particle propagator},
    description={a mathematical object (specifically, a Green's function) that dictates the 
    evolution of a single-particle state per the Sch\"odinger equation}
}

\newglossaryentry{LDM}
{
    name={LDM},
    description={Liquid Drop Model. An early, physically-intuitive model that treats the
    nucleus as a charged drop of nuclear fluid}
}

\newglossaryentry{finite-size corrections}
{
    name={finite-size correction},
    description={Any of a series of corrections that account for the
    non-zero size of scattering targets used in cross section measurements,
    including flux attenuation, angular uncertainty in the scattering path, and
    multiple scattering}
}
