\makeglossaries

\newglossaryentry{DOM}
{
    name={DOM},
    description={The \textbf{D}ispersive \textbf{O}ptical \textbf{M}odel, a phenomenological
optical-model framework for extracting nuclear structure and reaction information from experimental data.},
    sort={DOM}
}

\newglossaryentry{optical potential}
{
    name={optical potential},
    description={A complex potential used to approximate the microscopic nuclear
    many-body problem. Incident nucleons scatter off the potential in analogy
    to the refraction and absorption of light in optical media.
    In both cases, the real component of the potential determines
    elastic scattering, and the imaginary component determines inelastic
    scattering.},
    sort={DOM}
}

\newglossaryentry{nucleon self-energy}
{
    name={nucleon self-energy},
    description={A complex, non-local mathematical object that describes the 
    the interaction of a nucleon with another body (typically a nucleus) with an
    infinite sum of relevant Feynman diagrams. If the nucleon self-energy is known in a
    given system, a multitude of other important physics quantities
    (scattering amplitudes, the mean free path, the level density) can be
    extracted. The DOM asserts that the optical
    potential and the nucleon self-energy are the same object, enabling a
    phenomenological approach to solving the nuclear many-body problem.
    See \Gls{Dyson Equation}, \Gls{optical potential}, \Gls{DOM}},
    sort={DOM}
}
