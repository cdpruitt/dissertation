
\makeglossaries

\newglossaryentry{DOM}
{
    name={DOM},
    description={The \textbf{D}ispersive \textbf{O}ptical \textbf{M}odel, a phenomenological optical-model framework for extracting nuclear structure and reaction information from experimental data.},
    sort={DOM}
}

\newglossaryentry{optical potential}
{
    name={optical potential},
    description={A complex potential used to approximate the microscopic nuclear many-body problem. Incident nucleons scatter off the potential in analogy to the refraction and absorption of light in optical media.  In both cases, the real component of the potential determines elastic scattering, and the imaginary component determines inelastic scattering.},
    sort={DOM}
}

\newglossaryentry{nucleon self-energy}
{
    name={nucleon self-energy},
    description={A complex, non-local mathematical object that describes the interaction of a
        nucleon with another body (typically a nucleus) via an infinite sum of relevant Feynman 
        diagrams. If the nucleon self-energy is known in a given system, a multitude of other 
        important physics quantities (scattering amplitudes, the mean free path, the level density) 
        can be extracted. The DOM asserts a link between the optical potential and the nucleon self-
        energy through the subtracted dispersion relation \ref{SubtractedDispersion}, enabling 
        phenomenological constraints on solutions of the nuclear many-body 
        problem. See \Gls{Dyson Equation}, \Gls{optical potential}, \Gls{DOM}},
    sort={DOM}
}

\newglossaryentry{Dyson Equation}
{
    name={Dyson Equation},
    description={A self-consistent equation that relates the dressed propagator $G$ to the free
        propagator $G_{0}$ and the \Gls{nucleon self-energy}, $\Sigma$:
        \begin{equation}
            G = G_{0} + G_{0} \Sigma G
        \end{equation}
        The equation can be expressed pictorally via Feynman diagrams (shown in
        \ref{DysonEquation}).},
    sort={DOM}
}
