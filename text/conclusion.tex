\section{Neutron \tot results and outlook}
We have presented a new program of isotopic \tot measurements that provide critical data for
constraining the isovector strength of the nuclear potential at positive energies. By using
digitizer technology to reduce the processing time associated with each detection event by a factor
of ten, we were able to accumulate sufficient statistics far more quickly than in previous
measurements that used analog signal processing. Consequently, we were able to measure the \tot on
several rare isotopes up to 450 MeV, an energy range that was previously inaccessible.

%\begin{figure}
%  \begin{center}
%\includegraphics[width = 1.0\textwidth]{QuadrupoleCoulexAlignment.png}
%\caption{ Figure 13 from Ref. \cite{Olliver2003}. Decay angular distribution of $\gamma$-rays in the laboratory frame for quadrupole coulomb excitation ($J_i \rightarrow J_f = 2$) to different excitation energies of $^{56}$Ni, for $^{56}$Ni scattering off $^{209}$Bi at $E/A = 85$ MeV.}\label{Olliver}
%\end{center}
%\end{figure}

%\subsection{Coulomb Excitation at Intermediate Energies}

