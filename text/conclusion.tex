\section{Implications of experimental \tot\ and \el\ results}
We have presented a new program of isotopic \tot\ measurements that provide critical data for
constraining the isovector strength of the nuclear potential at positive energies. By using
digitizer technology to reduce the processing time associated with each detection event by a factor
of ten, we were able to accumulate sufficient statistics far more quickly than in previous
measurements that used analog signal processing. Consequently, we were able to measure the \tot\ on
several rare isotopes up to 450 MeV, an energy range that was previously inaccessible.

\section{Implications of DOM Results for Nuclear Structure}

\section{Topics for Future Study}
\tot on Fe and Cd isotopes (cf. with Ni and Sn isotope studies), \tot on all stable Sn isotopes
Proton \rxn studies on O, Ni, and Sn isotopes
Covariance analysis of DOM parameters
Publication of the DOM codebase
Isotope chain calculations (\snTwelve to \snFour), as was done with local DOM
Extension to odd-even and odd-odd nuclei
Extraction of global optical model (cf. Koning-Delaroche)
