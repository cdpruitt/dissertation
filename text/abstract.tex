\thispagestyle{plain}
\begin{center}

    ABSTRACT OF THE DISSERTATION

    Isotopically-Resolved Neutron Cross Sections as
    Probe of the Nuclear Optical Potential

    \vspace{0.5 cm}

    by

    \vspace{0.2 cm}

    Cole Davis Pruitt

    \vspace{0.2 cm}

    Doctor of Philosophy in Chemistry

    \vspace{0.2 cm}

    Washington University in St. Louis, 2019

    \vspace{0.2cm}

    Professor Lee Sobotka, Chairperson
\end{center}

\vspace{1cm}

Neutron scattering experiments provide direct access to the forces experienced by nucleons in the
nuclear environment. Due to the experimental difficulty of cross section
measurements with neutrons, isotopically-resolved neutron scattering cross sections are sorely
needed as inputs for many nuclear models.
This dissertation presents the results from a campaign of
isotope-specific neutron total cross section measurements on \oSixEight, \niEightFour,
\snTwelveFour, and \rhThree\ from
3-400 \mega\electronvolt\, and elastic scattering differential cross section measurements on
\snTwelveNatFour\ at 11 and 17 \mega\electronvolt\. Equipped with these new data and
with computational improvements to the Dispersive Optical Model (DOM),
we present DOM treatments of \oSixEight, \caAughtEight, \niEightFour, 
\snTwelveFour, and \pbEight. From these analyses across the nuclear chart, we place additional 
constraints on the neutron-proton asymmetry-dependence of nuclear properties, extract essential bound-state 
quantities including spectroscopic factors and neutron skins, and identify experimental data types
most useful for further enhancing our understanding of nuclear structure.

%\afterpage{\blankpage}
