\section{Event identification}
During production, runs were taken in batches of three, one each with \snTwelve,
\snFour, and the blank samples. Because the distance to each time-of-flight detector
was already measured, the time-of-flight for elastically-scattered neutrons could
be directly calculated and used to determine the delay from electronics and cabling.
The timestamp of each event was thus adjusted by a fixed amount
so that the first peak of the neutron spectrum aligned the expected
time-of-flight. Next, background $\gamma$-ray events were separated from relevant
neutron events by a pulse-shape discrimination analysis, shown in Fig. \ref{PHPSDPlot}.

\begin{figure}[ht]
    \includegraphics[width=0.9\textwidth]{figures/PHPSDPlot.png}
    \caption[Event pulse height (PH) vs. pulse-shape-discrimination (PSD) for
    a typical run]
    {
        Event pulse height (PH) vs. pulse-shape-discrimination (PSD) for
        a typical run. A gate (dashed line) isolates neutron events, which are
        used for subsequent analysis. At low pulse heights, the PSD output from the
        MPD-4 module is non-linear, making neutron-$\gamma$-ray separation more difficult
        (bottom-left of the figure).
    }
    \label{PHPSDPlot}
\end{figure}

Neutron events surviving this PSD gating are shown for a few typical runs in Fig. 
\ref{tiledRunData}. The first and second subfigures in the left column of the figure are from 
the same detector and arm angle, but the first is from a blank-sample run and
the second is from a \snFour\ sample-run. In each subfigure, the expected location
of the time-of-flight for elastic scattering on \snFour\ is marked with a dark blue arrow,
and the approximate FWTM time
resolution of the detector is marked with blue dashed lines. The expected
location of the time-of-flight peak for elastic scattering on atmospheric
N$_{2}$ is marked with a green arrow. The additional counts in the 55-57 ns region in
the second subfigure (compared to the first) are from elastic scattering on
\snFour. Energy-dependent detector efficiencies for each detector were provided by TUNL
and are shown above each histogram. For clarity, these efficiencies are plotted
relative to the efficiency of at the elastically-scattered neutron energy.

\begin{figure}[ht!]
    \centering
    \includegraphics[width=0.9\textwidth]{figures/tiledRunData.png}
    \caption[Histograms from typical runs showing neutron elastic scattering peak]
    {
        Neutron-event histograms for three typical runs, one with the blank
        sample and two with the \snFour\ sample.
        The expected location of the \snFour\ elastic scattering peak (dark blue arrow)
        and first two inelastic scattering peaks (light blue arrows) show an
        increased number of counts when the sample is in-beam (runs 253 and
        254). For reference, the expected location of the nitrogen elastic scattering
        peak is shown as green arrows and the expected location of the first- and
        second-inelastic scattering peaks for \snFour\ are shown as red arrows.
        The FWTM detector resolution is indicated by the blue dashed lines.
    }
    \label{tiledRunData}
\end{figure}

After scaling histogram counts by detector efficiency, histograms were
normalized by the total neutron flux
(i.e., total counts in the CMON detector for that run) and summed by detector
angle. Then, blank-run histograms
were subtracted from the isotopic-run histograms to yield the neutron scattering
events from the isotopic samples. Figure \ref{tiledAngleData} provides example 
results. As in the previous figure, dark blue arrows mark the anticipated
time-of-flight of the elastic scattering peak, green arrows mark elastic
scattering from atmospheric nitrogen, and blue dashed lines mark the anticipated
FWTM of the elastic scattering peak. Inelastic scattering peaks from the first and
second excited states samples are marked with light blue
arrows. For the 4M detector and at forward angles for both detectors, the elastic and first 
inelastic scattering peaks are closer in time and cannot be cleanly resolved. Measurements in this
kinematic regime are the most challenging as the increased overlap between
these peaks increases the uncertainty of the number of counts in the elastic
peak. We fit the amplitudes of two Gaussian distributions to the elastic and
first-inelastic peaks while fixing
the width and centroid of each Gaussian according to our time-of-flight resolution and
the expected time-of-flight. The integral of the first Gaussian provides the
number of counts in the elastic peak.

\begin{figure}[ht]
    \centering
    \includegraphics[width = 0.9\textwidth]{figures/tiledAngleData.png}
    \caption[Scaled event histograms showing neutron elastic scattering peak]
    {
        Scaled event histograms showing neutron elastic scattering peak. The
        neutron event histograms for isotopic-sample runs (dark gray),
        blank-sample runs (light gray), and the difference (in red) are plotted.
        The dark blue arrows indicate the expected location of the elastic scattering
        peaks on \snTwelveFour, and the light blue arrows indicate the first-
        and second-excited inelastic scattering peaks on \snTwelveFour. The
        green arrows indicate the expected location of the elastic scattering
        peaks on atmospheric N$_{2}$. A double-Gaussian function was fit to the
        subtracted histogram and the results of the fit drawn in dark
        blue (elastic peak) and light blue (first-inelastic peak).
    }
    \label{tiledAngleData}
\end{figure}

\section{Normalization}
To normalize the cross sections, the absolute neutron flux must be known.
To determine the absolute neutron flux, reference runs were taken using a graphite,
a polyethylene, and a blank sample at both the beginning and end of the experiment.
Events from one batch of these runs are
histogrammed in Fig. \ref{polyethyleneRef} for the 4M and 6M detectors.
The same conventions from Fig.
\ref{tiledAngleData} are used, except that now the light blue arrows correspond to
the elastic and first-excited states of \cTwelve. The dark gray histogram (back
histogram layer) shows the elastic and first-excited states of \cTwelve
and also a broad peak from elastic scattering on H. After scaling for the number
of moles in the graphite and polyethylene samples, stoichiometry, and neutron
flux, the graphite spectrum (light gray, middle histogram later) is subtracted from the polyethylene
spectrum, yielding neutron events from elastic scattering on protons (area of
the red histogram between the blue dashed lines). Given the
well-established n(p,p)n cross section, the ratio of absolute neutron flux to
the number of CMON counts was calculated:

\begin{equation}
    [insert monitor scaling equation]
\end{equation}

\noindent
We used the Scattering Analysis Interactive Database (SAID) code \cite{SAIDCode}
to generate
the proton-neutron elastic scattering cross sections needed in the above
equation. This flux-to-monitor ratio was then used to normalize the \snTwelveFour\
results.

\begin{figure}[ht!]
    \includegraphics[width = 0.9\textwidth]{figures/polyethyleneRef.png}
    \caption[Reference runs of neutron scattering on C and (CH$_{2}$)$_{n}$]
    {
        Reference runs of neutron scattering on C and (CH$_{2}$)$_{n}$
        The (CH$_{2}$)$_{n}$ run (back histogram, in dark gray) and C run
        (middle histogram, in light gray) are scaled by beam flux and number of
        atoms in each sample. Their difference (front histogram, in red) shows a
        broad peak corresponding to neutron-proton elastic scattering. The
        expected time-of-flight for this peak is shown by the blue arrow. The light blue arrows 
        indicate the expected position of the elastic and first-excited peaks
        from scattering on C. The green arrow indicates the expected position of
        the elastic peak from scattering on atmospheric N$_{2}$.
    }
    \label{polyethyleneRef}
\end{figure}

\section{Cross Section Calculation}

The absolute \el\ were calculated as:

\begin{equation} \label{ECSCalculation}
    \el = \el_{ref,\theta}
    \times \frac{X_{sample}}{X_{ref}} \times
    \frac{N_{sample}}{N_{ref}}
\end{equation}

\noindent
where $\el_{ref}$ is the (n,p) cross section at lab angle $\theta$, $X_{sample}$
and $X_{ref}$ are number of counts in the elastic scattering peak for the sample
and the reference runs, respectively (see Figs. \ref{tiledAngleData} and
\ref{polyethyleneRef}), and $N_{sample}$ and $N_{ref}$ are the number of atom in
the sample of interest and the number of hydrogen atoms in the polyethylene
sample, respectively.

\section{Finite Size Corrections}
In an idealized differential cross section measurement, the sample and neutron
detectors can be treated as point objects. In reality, the neutron beam,
samples, and detectors occupy a finite size, leading to so-called finite-size
effects that distort the measured cross sections. The experimenter is responsible
for applying appropriate corrections to make results size- and
apparatus-independent. The finite-size analysis for
a similar TUNL-based neutron \el\ measurement on $^{116,120}$Sn is described in detail
in \cite{GussPhDThesis}. In their analysis, Monte Carlo simulations using the
EFFIGY code were
prepared to generate a correction for geometric uncertainty of the neutron
scattering track, the possibility of multiple scattering
in the samples, and flux attenuation in the sample. These effects are illustrated in
\ref{GussFiniteSizeDiagram}.
For the isotopic Ni and Sn samples they studied, they generated finite-size
corrections on the order of 1-10\% depending on the scattering angle. As seen in
Fig. \ref{GussFiniteSizeEffect}, the
biggest effect is on the depth of the diffraction minima.
However, their samples
were an order of magnitude larger than our
samples: 42.59 g and 44.73 g for their \snSixteen\ and \snTwenty\ samples,
respectively, compared to 4.97 g and 5.55 g for our \snTwelve\ and \snFour\
samples. Thus we anticipated a dramatically smaller correction would be
required.

\begin{figure}[ht!]
    \begin{center}
        \includegraphics[width = 0.9\textwidth]{figures/GussFiniteSizeDiagram.png}
        \caption[Illustration of finite-size effects relevant for \el cross
        section measurements]
        {
            Illustration of finite-size effects relevant for \el cross
            section measurements, from the PhD thesis of P. Guss
            \cite{GussPhDThesis}. Due to the uncertainty in the exact path taken
            by neutrons during scattering, a series of finite-size corrections
            must be applied to recover the true cross section.
        }
        \label{GussFiniteSizeDiagram}
    \end{center}
\end{figure}

\begin{figure}[ht!]
    \begin{center}
        \includegraphics[width = 0.9\textwidth]{figures/GussFiniteSizeEffect.png}
        \caption[Effect of finite-size corrections on previous neutron \el\ measurement at
        TUNL]
        {
            Effect of finite-size corrections on previous neutron \el\ measurement at
            TUNL, from the PhD thesis of P. Guss \cite{GussPhDThesis}.
            The raw data from this measurement on Ni isotopes are shown as data
            points. Using a Monte Carlo simulation, the authors of the previous
            study generated a correction accounting for multiple scattering in
            their samples, the angular uncertainty stemming from the volume
            of their samples, resulting in the dashed curve. With beam
            attenuation also considered, the cross section is uniformly
            increased across the angular range, giving the solid curve, which
            they take to be the ``true'' cross section. Because our targets are
            approximately an order of magnitude smaller, we see far smaller
            finite size effects in our simulation (see Figs.
            \ref{FiniteSizeAngular} and \ref{FiniteSizeAttenuation}).
        }
        \label{GussFiniteSizeEffect}
    \end{center}
\end{figure}

Figure \ref{GussFiniteSizeDiagram} illustrates the possibility of multiple scattering in the samples
and the small degree of angular uncertainty in the neutron scattering path.
The effect of this angular uncertainty on the measured cross
sections is to ``wash out'' the sharp diffraction minima expected to be present in the true cross
section. To assess the magnitude of this effect from our
samples, an iterative simulation was prepared in which a uniform beam of neutrons
impinged on the sample volume and was scattered into the time-of-flight
detectors. To select each neutron's scattering angle in the simulation, we took the
raw cross section from our measurement to be the true cross section. After
scattering, neutrons were scored in simulated detectors with the same the dimensions
used in the real experiment and an ``output'' cross section was generated. The
output is thus a weighted convolution of the input cross section
over the finite size effects of the samples and detectors. In addition to
running a simulation with the sample sizes used in the experiment, we performed
addition simulations with exaggerated sizes for the
sample to make finite-size effects more visible. A comparison between the input
and output cross sections shows the effects of beam attenuation and angular
uncertainty, seen in Figs. \ref{FiniteSizeAttenuation} and \ref{FiniteSizeAngular}.
[more comments about finite size w/r/t Guss et al.]
To calculate correction factors, we divided the simulation's
input cross section by the output cross section for each angle and multiplied our
experimental results by this factor. In principle, this procedure to generate
the correction should be repeated iteratively, with the new corrected cross
section plugged back into the simulation, but in practice the corrections were so small
that only the first iteration was required.

\section{Results}
\subsection{\snTwelveFour\ \el\ at 11 MeV}
Absolute cross sections 

\begin{figure}
    \begin{center}
        \includegraphics[width = 0.9\textwidth]{figures/neutronECS_Sn_11MeV.png}
        \caption{Neutron \el cross sections on $^{112,124}$Sn at 11
    MeV: our results and literature data}
    \label{SnECS_11MeV}
\end{center}
\end{figure}

\subsection{\snTwelveFour\ \el\ at 17 MeV}

\begin{figure}
    \begin{center}
        \includegraphics[width = 0.9\textwidth]{figures/neutronECS_Sn_17MeV.png}
        \caption{Neutron \el cross sections on $^{112,nat,124}$Sn at 17
    MeV: our results and literature data} \label{SnECS_17MeV}
\end{center}
\end{figure}

\afterpage{\clearpage}
