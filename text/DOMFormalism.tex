The Dispersive Optical Model (DOM) is a phenomenological framework useful for 
extracting information about nuclear properties and structure from experimental
data.

The fundamental object of interest in the DOM is the \Gls{optical potential},
which is identified as the \Gls{nucleon self-energy}. This potential represents
the nuclear environment experienced
by a nucleon as it traverses the nucleus being represented. In general, the
potential is both non-local and complex, possessing both real (flux-conserving)
and imaginary (flux-removing) parts. The functions used to parameterize the optical potential
are selected to conform with general physical intuition about the nuclear
many-body problem and past experience with optical potentials throughout the
field. For example, the most important term, the Hartree-Fock (HF) potential
that binds the nucleus together, is defined by a classic Woods-Saxon form to
which non-locality has been added: [insert volume term formula]. The imaginary
potential is separated into a Woods-Saxon volume component and a
Woods-Saxon-derivative surface component. Each of these subcomponents has
a non-linear energy dependence reflecting our expectation that surface-like
behaviors should arise around 10-20 MeV and volume-like behavior should dominate
above 50 MeV, with a mixture of the two in the intermediate region.

Beyond these basic considerations, several additional features make the 
DOM a useful tool. First is the enforcement of a
dispersion relation (the Kramers-Kronig relations) between the real and imaginary
halves of the potential, ensuring that the potential is causal (i.e., the
time-ordering of the operator representation of the self-energy is maintained).

\section{Definition of the single-particle propagator}
The central project of the Dispersive Optical Model (like any optical model) is
to understand how nucleons move about in a nuclear many-body system. Specifically,
we wish to know
how a nucleon with energy $E$ and quantum numbers $\alpha$
at time $t_{0}$ will be measured at a
later time $t$ with quantum numbers $\beta$ after its interaction with the
nuclear environment, which is modeled with an optical potential. Given the
Hamiltonian that incorporates this interaction, the Schr\"odinger equation
relates the Hamiltonian to the time evolution of this state:

\begin{equation}
    i\hbar\frac{\partial}{\partial t}\ket{\alpha, t_{0};t} = H\ket{\alpha,
    t_{0}; t}
\end{equation}

where $\ket{\alpha, t_{0},t}$ is the state at time $t$, given an initial state
$\ket{\alpha, t_{0}}$. Simple substitution shows that this initial state
propagates in time according to:

\begin{equation}
    \ket{\alpha, t_{0}, t} = e^{-\frac{i}{\hbar}H(t-t_{0})}\ket{\alpha, t_{0}}
\end{equation}

Concretely, given initial position quantum numbers $\boldmath{r}$, the wavefunction
of a state $\psi(\boldmath{r},t)$ is the sum of the contributions of 

DOM Parameterization

To parameterize the DOM's optical potential, we rely on standard functional
forms common in nuclear theory calculations.

The real part of the potential is comprised of a Hartree-Fock component and
a spin-orbit component (plus a Coulomb term if the projectile is a proton).
The Hartree-Fock component $V_{HF}$ has two subcomponents:

\begin{equation}
    V_{HF}(r,r') = V_{vol}(r,r') + V_{WB}(r)
\end{equation}

The non-local Hartree-Fock volume term $V_{vol}(r,r')$, is defined as
a Woods-Saxon form coupled to a Gaussian non-locality:

\begin{equation}
    V_{vol}(r,r') =
    \dfrac{-V}{1+e^{(r-R)/a}}\cdot\dfrac{1}{\pi^{\frac{3}{2}}\beta^{3}}
    \exp{\frac{|r-r'|^{2}}{\beta^{2}}}
\end{equation}

where $V$, $R$, and $a$ are the depth, radius, and diffuseness of the HF potential,
and $beta$ is the extent of the non-locality. The local Hartree-Fock wine-bottle
term $V_{wb}(r)$, named for resemblence to the dimple at the bottom of a wine
bottle, is defined as a Gaussian centered at the nuclear origin:

\begin{equation}
    V_{wb}(r) = V\exp{\frac{r^{2}}{\sigma^{2}}}
\end{equation}

The real spin-orbit component $V_{so}$
is defined using a derivative-Woods-Saxon shape to
accommodate the expectation that the spin-orbit coupling is strongest near the
nuclear surface:

\begin{equation}
    V_{so}(r,r') =
    \frac{d}{dr} \frac{-V}{1+e^{\frac{(r-R)}{a}}}\cdot\frac{1}{\pi^{\frac{3}{2}}\beta^{3}} \exp{\frac{|r-r'|^{2}}{\beta^{2}}}
\end{equation}

The imaginary part of the potential is comprised of independent surface and volume terms
both above and below the Fermi surface, plus an imaginary spin-orbit term:

\begin{equation}
    V_{HF}(r,r') = V_{vol}(r,r') + V_{WB}(r)
\end{equation}

The occupation number of nucleons with quantum numbers $\alpha$ can be
calculated directly from the imaginary component of the single-particle
propagator:

\begin{eqnarray}
    n(\alpha)
    & = & \braket{\Psi^{N}_{0}|a^{\dagger}_{\alpha}a_{\alpha}|\Psi^{N}_{0}}\\
    & = & \sum_{n}|\braket{\Psi^{N-1}_{n}|a_{\alpha}|\Psi^{N}_{0}}|^{2}\\
    & = & \int_{-\infty}^{\epsilon_{F}^{-}} dE
\sum_{n}|\braket{\Psi^{N-1}_{n}|a_{\alpha}|\Psi^{N}_{0}}|^{2}
\delta(E-(E^{N}_{0}-E^{N-1}_{n}))\\
& = & \int_{-\infty}^{\epsilon_{F}^{-}} dE \frac{1}{\pi}\operatorname{Im}G(\alpha,\alpha;E)
\end{eqnarray}



For nuclei with open subshells (e.g., the $\upnu$ 0\dFive in $^{18}$O and
$\upnu$ 0\fFive in $^{58}$Ni), an additional
pairing parameter $\Delta$ was added to
account for these subshells' fractional occupatio $n_{\pm}$ of the open subshells. due to pairing
effects. This parameter splits partially-occupied subshells (e.g., the $\upnu$\dFive
for \oEight) into upper and lower sublevels with energies $E_{\pm}$:

\begin{equation}
    E_{\pm} = \mu \pm ((\epsilon_{F}-\mu)^{2} + \Delta^{2})^{frac{1}{2}}
\end{equation}

where $\mu$ is the energy of the open subshell before pairing is considered and
$\epsilon_{F}$ is the Fermi
energy. The magnitude of $\Delta$ corresponds to the energy difference between
adding/removing a nucleon to/from that subshell. The subshell's particle
capacity is split between the upper and lower sublevels, and only the :

\begin{equation}
    n_{\pm} = \frac{1}{2}\left( 1-\frac{\chi}{E_{\pm}}\right)
\end{equation}

where $\chi = |E_{\pm}-\mu| - (\epsilon_{F} - \mu)$. Only occupation in the
lower sublevel is counted toward the total particle number. For each open-shell
nucleus with nucleon numbers $N, Z$ , the pairing gap $\Delta$ was fixed according to:

\begin{equation}
    \Delta(N,Z) = \frac{1}{4}\left(B(N-2,Z)-3B(N-1,Z) + 3B(N,Z)-B(N+1,Z)\right)
\end{equation}

where $B(N,Z)$ is the nuclear binding energy.

-> insert table for o18, ni58, ni64, sn112, sn124

%\begin{figure}
%  \begin{center}
%\includegraphics[width = 0.9\textwidth]{CAD_RUSS2.pdf}
%\caption{A 3D CAD model of the two annular Si detectors used in the experiment.} \label{RUSS2}
%\end{center}
%\end{figure}

%\begin{table}
%  \begin{center}
%    \caption{Calibration beams and the energies generated with the degraders.}\label{CBeams}
%  \begin{tabular}{ccccc}
%    \hline \hline
%    Species & Energy & Target & Thickness & Degraded Energy  \\ 
%            & [MeV/A] & &[mg/cm$^2$] & [MeV/A] \\
%     \hline
%    $p$ & 24.2 &  Au & 20.0 & 24.0   \\
%           &  & Al & 429 & 15.8 \\
%    \hline
%    $d$ & 24.2 &  Au &20.0 & 24.1 \\
%            & & Al & 429 &20.3 \\
%    & &  Al & 858 & 15.8 \\
%            & 12.0 &  Au & 20.0 & 11.9 \\
%    \hline
%    $\alpha$ & 24.0 &  Au & 20.0 & 23.8 \\
%     & & Al & 429 &15.6 \\
%    \hline \hline
%  \end{tabular}
%\end{center}
%\end{table}

\afterpage{\clearpage}
