The Dispersive Optical Model (DOM) is a phenomenological framework useful for 
extracting information about nuclear properties and structure from experimental
data.

The fundamental object of interest in the DOM is the *optical potential*, or
*self-energy*. This quantity represents thei nuclear environment experienced
by a nucleon as it traverses the nucleus under investigation. In general, the
potential is both non-local and complex, possessing both real (flux-conserving)
and imaginary (flux-removing) parts. The functions used to parameterize the optical potential
are selected to conform with general physical intuition about the nuclear
many-body problem and past experience with optical potentials throughout the
field. For example, the most important term, the Hartree-Fock (HF) potential
that binds the nucleus together, is defined by a classic Woods-Saxon form to
which non-locality has been added: [insert volume term formula]. The imaginary
potential is separated into a Woods-Saxon volume component and a
Woods-Saxon-derivative surface component. Each of these subcomponents has
a non-linear energy dependence reflecting our expectation that surface-like
behaviors should arise around 10-50 MeV and volume-like behavior should arise
above around 100 MeV, with a mixture of the two in the intermediate region.

Beyond these basic considerations, several additional features make the current
DOM advanced by our group a uniquely useful tool. First is the enforcement of a
dispersion relation between the real and imaginary halves of the potential,
the Kramers-Kronig relation. 

%\begin{figure}
%  \begin{center}
%\includegraphics[width = 0.9\textwidth]{CAD_RUSS2.pdf}
%\caption{A 3D CAD model of the two annular Si detectors used in the experiment.} \label{RUSS2}
%\end{center}
%\end{figure}

%\begin{table}
%  \begin{center}
%    \caption{Calibration beams and the energies generated with the degraders.}\label{CBeams}
%  \begin{tabular}{ccccc}
%    \hline \hline
%    Species & Energy & Target & Thickness & Degraded Energy  \\ 
%            & [MeV/A] & &[mg/cm$^2$] & [MeV/A] \\
%     \hline
%    $p$ & 24.2 &  Au & 20.0 & 24.0   \\
%           &  & Al & 429 & 15.8 \\
%    \hline
%    $d$ & 24.2 &  Au &20.0 & 24.1 \\
%            & & Al & 429 &20.3 \\
%    & &  Al & 858 & 15.8 \\
%            & 12.0 &  Au & 20.0 & 11.9 \\
%    \hline
%    $\alpha$ & 24.0 &  Au & 20.0 & 23.8 \\
%     & & Al & 429 &15.6 \\
%    \hline \hline
%  \end{tabular}
%\end{center}
%\end{table}

\afterpage{\clearpage}
