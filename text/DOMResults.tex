As with any high-dimensional optimization problem, the fitter must be vigilant
against the overfitting of data. In practice, this requires:

parsimony with the number of parameters used in the model

common-sense checking "under the hood" of the optimization to verify that
parameter values make sense given the assumptions that undergird the model

understanding of what the value function is (that is, the function being
minimized/maximized) and whether it needs to be changed

cross-correlation between parameters to understand the relationship between
parameters and the effect that each has on predictions made using the model


TCS affected by HF parameters, spin orbit, imaginary above
RCS affected by imaginary above, HF parameters
ECS affected by HF parameters, spin orbit, imaginary above
APower affected by HF parameters, spin orbit,  imaginary above
Charge Density affected by HF parameters, imaginary below
Levels affected by HF parameters, spin orbit,
Spectral functions affected by HF parameters, imaginary below
RMSRadii affected by HF parameters, imaginary below

%\begin{figure}
%  \begin{center}
%\includegraphics[width = 0.9\textwidth]{CAD_RUSS2.pdf}
%\caption{A 3D CAD model of the two annular Si detectors used in the experiment.} \label{RUSS2}
%\end{center}
%\end{figure}

%\begin{table}
%  \begin{center}
%    \caption{Calibration beams and the energies generated with the degraders.}\label{CBeams}
%  \begin{tabular}{ccccc}
%    \hline \hline
%    Species & Energy & Target & Thickness & Degraded Energy  \\ 
%            & [MeV/A] & &[mg/cm$^2$] & [MeV/A] \\
%     \hline
%    $p$ & 24.2 &  Au & 20.0 & 24.0   \\
%           &  & Al & 429 & 15.8 \\
%    \hline
%    $d$ & 24.2 &  Au &20.0 & 24.1 \\
%            & & Al & 429 &20.3 \\
%    & &  Al & 858 & 15.8 \\
%            & 12.0 &  Au & 20.0 & 11.9 \\
%    \hline
%    $\alpha$ & 24.0 &  Au & 20.0 & 23.8 \\
%     & & Al & 429 &15.6 \\
%    \hline \hline
%  \end{tabular}
%\end{center}
%\end{table}

\afterpage{\clearpage}
