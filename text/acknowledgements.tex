Being a member of the Radiochemistry group at Washington University has defined
my experience in graduate school. Demetrios Sarantites, Jon Elson,
Walter Reviol, Bob Charity, and -- especially -- my advisor Lee Sobotka have spent
countless hours mentoring me over the last five years, encouraging my successes
and spurring my development as a scientist. My mental model of the nucleus derives from
their (far deeper) knowledge. I can only hope to repay
their kindness by sharing their insights with future students and colleagues.

Fellow Radiochemistry graduate students Kyle Brown, Tyler Webb, Dan Hoff, and Dan Mulrow,
have been both great friends and capable colleagues at the lab.
Their emotional and social support have been an important part of my
growth throughout graduate school.

My collaboration with the Washington University Theoretical Nuclear Physics group has
been an enriching one. The DOM results presented in this
work are a testament to the patience and intellect of Wim Dickhoff and 
the assistance of Physics graduate students Natalia Calleya,
Hossein Mahzoon, and especially Mack Atkinson.

Many faculty and staff at LANSCE and TUNL were critical for the 
experimental measurements detailed in this work. In particularly, I want to thank Matt
Devlin, Hye Young Lee, Shea Mosby, and Nik Fotiadis at LANSCE and Calvin Howell
and Ron Malone at TUNL. Without their abundant know-how and generous
assistance, our experiments could not have succeeded.

Lastly, I am forever grateful to my family, especially my parents, for
kindling my curiosity as a child and for supporting my pursuits wherever they
lead.

\vspace{20pt}

\begin{flushright}
  Cole D. Pruitt
\end{flushright}

\textit{Washington University in St. Louis}

\textit{May 2019}

\clearpage
