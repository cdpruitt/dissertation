\section{Sample Preparation}

The isotopic Sn samples were prepared by melting isotopically-enriched foils to
800 C in a tube furnace, cooling to ambient temperature, and pressing to the
desired shape in a tempered die. To reduce formation of tin
oxide during melting, the samples were placed in a vitreous carbon crucible
and kept under a reducing atmosphere (90% Argon/ 10%H2) while at elevated
temperatures [\cite photo]. Of the 4.9 grams of Sn112 used to prepare the sample,
3.5 grams were from
semi-permanent loan from the [INSERT GROUP NAME] from LBNL and the remaining
Sn112 purchased from [ISOTOPE MANUFACTURER]. All of the 5.8 grams of Sn124 were purchased from
[ISOTOPE MANUFACTURER]. The natural Sn sample was prepared by melting and
pressing pellets [MANUFACTURER]. Loss of isotopic material during the
manufacturing process was minimal [insert table citation] and the final 
shapes and densities of the samples are within [X%] of the desired values.

To conduct the total cross section measurements on oxygen isotopes,
isotopically-enriched water samples were prepared (a technique also used by
\cite{previous O18 results}; cf.  with ZnO and BeO results of \cite{Finlay}).
Because the natural abundance of O16 in H2O is >99%, a sample of
ordinary distilled water was used to make the O16 measurement. For the O18
measurement, we used distilled water enriched to >99% in O18 from
[manufacturer]. Both samples were enclosed in brass vessels with very thin
(~0.001") brass endcaps, to minimize attenuation. At the temperature and
pressure of the experimental facility, the amount of dissolved gas and ions
in the water samples were small enough to have no effect on the measurement.

The natural and isotopic Ni samples were prepared by [insert name] at Oak Ridge National Lab 
(ORNL) to match the diameter of our Sn samples.

In addition, two natural-abundance graphite samples of different lengths and one
natural Pb sample, all with the same diameter as the Sn and Ni samples, were
prepared by the Washington University Machine Shop. These
samples were used to benchmark the total cross sections measured at LANSCE by
comparison with previous measurements of the C and Pb total cross sections
available in the literature (\cite Abfalterer and Finlay). Before use any
measurement, the C samples were baked in an oven for several
hours to remove residual machining oil and water.

%\begin{figure}
%  \begin{center}
%\includegraphics[width = 0.9\textwidth]{CAD_RUSS2.pdf}
%\caption{A 3D CAD model of the two annular Si detectors used in the experiment.} \label{RUSS2}
%\end{center}
%\end{figure}

%\begin{table}
%  \begin{center}
%    \caption{Calibration beams and the energies generated with the degraders.}\label{CBeams}
%  \begin{tabular}{ccccc}
%    \hline \hline
%    Species & Energy & Target & Thickness & Degraded Energy  \\ 
%            & [MeV/A] & &[mg/cm$^2$] & [MeV/A] \\
%     \hline
%    $p$ & 24.2 &  Au & 20.0 & 24.0   \\
%           &  & Al & 429 & 15.8 \\
%    \hline
%    $d$ & 24.2 &  Au &20.0 & 24.1 \\
%            & & Al & 429 &20.3 \\
%    & &  Al & 858 & 15.8 \\
%            & 12.0 &  Au & 20.0 & 11.9 \\
%    \hline
%    $\alpha$ & 24.0 &  Au & 20.0 & 23.8 \\
%     & & Al & 429 &15.6 \\
%    \hline \hline
%  \end{tabular}
%\end{center}
%\end{table}

\afterpage{\clearpage}
