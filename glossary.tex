\makeglossaries

\newglossaryentry{DWBA}
{
    name=DWBA,
    description={\textbf{D}istorted \textbf{W}ave \textbf{B}orn \textbf{A}pproximation: A method for calculating scattering transition amplitudes to describe nuclear reactions, as well as other quantum-mechanical scattering phenomena},
    sort={DWBA}
}
 
\newglossaryentry{DOM}
{
    name=DOM,
    description={\textbf{D}ispersive \textbf{O}ptical \textbf{M}odel, which uses dispersive corrections to consolidate nuclear structure and reactions},
    sort={DOM}
}
 
\newglossaryentry{GEANT}
{
  name=GEANT,
  description={\textbf{GE}ometry \textbf{AN}d \textbf{T}racking toolkit for the simulation of the passage of particles through matter. A CERN creation. Its areas of application include high energy, nuclear and accelerator physics, as well as studies in medical and space sci. [Nucl. Instru. Meth. A 506 (2003) 250-303, A 835 (2016) 186-225, and IEEE Trans. on Nucl. Sci. 53 No. 1 (2006) 270-278.] \n Can be downloaded http://geant4.cern.ch/}
}

\newglossaryentry{ROOT}
{
  name=ROOT,
  description={A modular scientific software framework. It provides all the functionalities needed to deal with big data processing, statistical analysis, visualisation and storage. It is mainly written in C++ but integrated with other languages such as Python and R. \n Can be Downloaded from https://root.cern.ch/}
}

\newglossaryentry{Inventor}
{
  name=Inventor,
  description={A 3-D CAD (Computer Added Design) software package. Can be Downloaded from https://www.autodesk.com/products/inventor/overview}
}

\newglossaryentry{PMT}
{
  name=PMT,
  description={\textbf{P}hoto\textbf{M}ultiplier \textbf{T}ube.  The most common type of light transducer used for scintillators. Following a photocathode of low work function. A series of dynodes with increasing positive voltage convert each cascading electron into several. The high-gain output is both linear and fast, reproducing the time characteristics of the scintillator. The other common type of light transducer is a PD with a solid-state version (SiPM) becoming popular. PMTs however remain the lowest noise and highest gain light transducer.  The other transducers are both smaller and magnetic field insensitive}
}

\newglossaryentry{ADC}
{
  name=ADC,
  description={\textbf{A}nalog-to-\textbf{D}igital \textbf{C}onverter. Historically The ADCs sat in the acquisition crate but most modern systems place them closer to the front-end signal processing electronics. For example, The HINP-16c systems employ ADCs on the CBs},
  sort={ADC}
}

\newglossaryentry{CFD}
{
  name=CFD,
  description={\textbf{C}onstant-\textbf{F}raction \textbf{D}iscriminator. A device that takes a linear input and produces a logical timing output independent of the amplitude of the input. It accomplishes this by finding the zero-crossing point of a waveform constructed from adding delayed and attenuated versions of the original input. Operation requires both a delay and an enabling threshold for a leading-edge discriminator (LED). CFD can be implemented in analog or digital electronics.  Using a LED along is a simpler, faster and less expensive way to produce a logical timing signal however the time of its output depends on the amplitude of the input signal. A (software) ”walk” correction can be applied to a LED to make the time amplitude independent if the pulse-height is available and if it is tolerable to make the correction, either in firmware or software, later in real time},
  sort={CFD}
}

\newglossaryentry{CSA}
{
  name=CSA,
  description={\textbf{C}harge-\textbf{S}ensitive \textbf{A}mplifier. They exist in all signal processing lines and are generally the resolution determining component},
  sort={CSA}
 }

  \newglossaryentry{GDG}
{
  name=GDG,
  description={\textbf{G}ate and \textbf{D}elay \textbf{G}enerator. These are implemented using  one of the standard IEEE single-ended (e.g. NIM or TTL) or differential (ECL or LVD) logic standards}
 }

  \newglossaryentry{TDC}
{
  name=TDC,
  description={\textbf{T}ime-to-\textbf{D}igital \textbf{C}onverter. A module for determing timing from a start and stop signal. Typically a capacitor is charged at a start time until the system receives a stop (called a time to charge converter i.e. a TCC). The charge on the capacitor can then be read out by an ADC. The time between start and stop can be determined from the amount of charge collected by the ADC},
  sort={TDC}
 }

 \newglossaryentry{TCC}
 {
     name=TCC,
     description={\textbf{T}ime-to-\textbf{C}harge \textbf{C}onverter. Creates a charge proportional to the time-difference between logical signals. The latter are usually produced by CFDs. The charges, temporary stored on capacitors, are sequenced into an ADC}
 }

\newglossaryentry{DSP}
{
  name=DSP,
  description={\textbf{D}igital \textbf{S}ignal \textbf{P}rocessing. The output of a CSA is digitized and all subsequent data processing works with these sequence of numbers}
 }

 \newglossaryentry{LANL}
{
  name=LANL,
  description={\textbf{L}os \textbf{A}lamos \textbf{N}ational \textbf{L}aboratory}
 }

 \newglossaryentry{LANCSE}
{
  name=LANSE,
  description={\textbf{L}os \textbf{A}lamos \textbf{N}eutron \textbf{S}cience \textbf{C}enter. This facility produces neutrons though the interaction of (up to) 800-MeV protons on tungsten. }
 }

 
 \newglossaryentry{TUNL}
{
  name=TUNL,
  description={\textbf{T}riangle \textbf{U}niversities \textbf{N}uclear \textbf{L}aboratory, on the campus of Duke University. This facility houses a tandem accelerator},
  sort={TUNL}
 }

  \newglossaryentry{WNR}
{
  name=WNR,
  description={\textbf{W}eapons \textbf{N}eutron \textbf{R}esearch facility at LANSCE}
 }

  \newglossaryentry{LDM}
{
  name=LDM,
  description={The \textbf{L}iquid \textbf{D}rop \textbf{M}odel, which treats nuclei as fluid of protons and neutrons},
  sort={LDM}
 }

  \newglossaryentry{NIM}
{
  name=NIM,
  description={\textbf{N}uclear \textbf{I}nstrumentation \textbf{M}odule, which are used for creating different electronics logic for running experiments},
  sort={NIM}
 }

  \newglossaryentry{FRIB}
{
  name=FRIB,
  description={\textbf{F}acility for \textbf{R}are \textbf{I}sotope \textbf{B}eams, which is currently under construction at Michigan State University and will be used to access nuclei far from the valley of stability},
  sort={FRIB}
 }

\newglossaryentry{GFMC}
{
  name=GFMC,
  description={\textbf{G}reen's \textbf{F}unction \textbf{M}onte \textbf{C}arlo: A method for performing quantum-mechanical calculations. This method has been used to predict many of the properties for $A < 12$ nuclei},
  sort={GFMC}
}
